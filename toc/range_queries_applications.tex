\subsection{Applications of the Range Queries Problem}\label{subseq:rmqapp}
There are many natural applications of the range queries problem for a~collection of records in a~database: computing the total population of cities that are at most some distance away from a~given point, computing an average salary in a~given period of time, finding the minimum depth on a~given subrectangle on a~sea map, etc. Below, we review some of the less straightforward applications where efficient algorithms for the range queries problem are usually combined with other algorithmic ideas.

\textbf{String algorithms and computational biology.}
It is possible to preprocess a~given string in $O(n)$ time (where $n$ is its length) so that to then find the longest common prefix of any two suffixes of the original string in constant time. This is done by first constructing the suffix array and the longest common prefix array of the string and then using an efficient RMQ algorithm.

\textbf{Computational geometry.} Algorithms for the range queries problems can be used together with the scanning line technique to solve efficiently various problems like: given a~set of segments on a~line, compute the number of intersecting pairs of segments; or, given a~set of rectangles and a~set of points on a~plane, compute, for each each rectangle, the number of points it contains.
